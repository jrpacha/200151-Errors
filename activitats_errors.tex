% %%%%%%%%%%%%%%%%%%%%%%%%%%%%%%%%%%%%%%%%%%%%%%%%%%%%%%%%%%%%%%%%%%%%%%%%%%%%
% Activitats Tema 1. Aritmètica Finita i Control d'Errors
% %%%%%%%%%%%%%%%%%%%%%%%%%%%%%%%%%%%%%%%%%%%%%%%%%%%%%%%%%%%%%%%%%%%%%%%%%%%%
\documentclass[slidetop,compress,mathserif,10pt]{beamer}
%\setbeamertemplate{theorems}[numbered]
\setbeamertemplate{caption}[numbered]

\usepackage{ragged2e}
\usepackage{amsmath,amssymb}
\usepackage{amsfonts}
\usepackage{mathtools}
\usepackage{enumerate}
\usepackage{wrapfig}
\usepackage[catalan]{babel}
\selectlanguage{catalan}
\usepackage[utf8]{inputenc}
\usepackage[rightcaption]{sidecap}

\usepackage{ifpdf}
\usepackage{graphicx}
\DeclareGraphicsExtensions{.eps}
\usepackage{subfigure}
\usepackage{shortlst}
\usepackage{tabu}
\usepackage{listings}
\lstset {%
  language=C++,
  showstringspaces=false
}

%-----------------------------------------------------------------------------
%pdflatex,
%Macros para producir hyperlinks
\DeclareGraphicsExtensions{.pdf,.jpg,.png}
\hypersetup{
  pdfauthor   = {Equip Docent ALN},
  pdftitle    = {Activitats Tema 1: Arim\`{e}tica Finita i Control d'Errors},
  pdfsubject  = {ALN 200151},
    pdfstartview = {},
    colorlinks  = {true},
    bookmarksnumbered = {true},
    hyperindex  = {true}
}
\pdfadjustspacing=1
\usepackage{url}

% THEOREMS -------------------------------------------------------------------
\newtheorem{thm}{Teorema}%[section]
\newtheorem{cor}[thm]{Coro\lgem ari}
\newtheorem{lem}[thm]{Lemma}
\newtheorem{prop}[thm]{Proposició}
\theoremstyle{definition}
\newtheorem{defn}[thm]{Definició}
\theoremstyle{remark}
\newtheorem{rem}{Remarca}
\newtheorem{ex}{Exemple}

\newenvironment{demo}
{\textsc{Demostració. }}{\quad \hfill $\blacksquare$}

% MATH -----------------------------------------------------------------------
\newcommand{\norm}[1]{\left\Vert#1\right\Vert}
\newcommand{\abs}[1]{\left\vert#1\right\vert}
\newcommand{\set}[1]{\left\{#1\right\}}
\newcommand{\bs}[1]{\ensuremath{\boldsymbol{#1}}}
\newcommand{\C}{\ensuremath{\mathbb{C}}}
\newcommand{\R}{\ensuremath{\mathbb{R}}}
\newcommand{\N}{\ensuremath{\mathbb{N}}}
\newcommand{\M}{\ensuremath{\mathbb{M}}}
\newcommand{\Z}{\ensuremath{\mathbb{Z}}}
\newcommand{\eps}{\ensuremath{\epsilon}}
\newcommand{\veps}{\ensuremath{\varepsilon}}
\newcommand{\To}{\longrightarrow}
\newcommand{\BX}{\mathbf{B}(X)}
\newcommand{\A}{\mathcal{A}}
\newcommand{\Or}{\mathcal{O}}

\newcommand{\argcosh}{\ensuremath{\mathrm{argcosh}}\,}
\newcommand{\argsinh}{\ensuremath{\mathrm{argsinh}}\,}
\newcommand{\argtanh}{\ensuremath{\mathrm{argtanh}}\,}
\newcommand{\argcoth}{\ensuremath{\mathrm{argcoth}}\,}

\newcommand{\rme}{\mathrm{e}}
\newcommand{\rmi}{\mathrm{i}}
\newcommand{\rmd}{\mathrm{d}}

\newcommand{\re}{\mathrm{Re\:}}
\newcommand{\im}{\mathrm{Im\:}}

\newcommand{\wt}[1]{\ensuremath{\widetilde{#1}}}

\DeclareMathOperator{\Hessian}{Hess}
% ----------------------------------------------------------------------------

\newcommand{\gl}{\guillemotleft}
\newcommand{\gr}{\guillemotright}
\newcommand{\sst}{\ensuremath{\scriptstyle}}
\renewcommand{\thefootnote}{\fnsymbol{footnote}} %{$\scriptstyle (*)$}

% Counters -------------------------------------------------------------------
\newcounter{saveenum}

%-----------------------------------------------------------------------------
\beamersetuncovermixins{\opaqueness<1>{25}}{\opaqueness<2->{15}}
\setbeamertemplate{caption}[numbered]
\setbeamerfont{caption}{size=\scriptsize}

\newenvironment<>{varblock}[2][\textwidth]{%
  \setlength{\textwidth}{#1}
  \begin{actionenv}#3%
    \def\insertblocktitle{#2}%
    \par%
    \usebeamertemplate{block begin}}
  {\par%
    \usebeamertemplate{block end}%
  \end{actionenv}}

\begin{document}
 \title[]{%
 {\Large ALN (200151)}\\
  {\large 1. Aritm\`{e}tica Finita i Control d'Errors}
 }

\date{}
\maketitle

%------------------------------- Activitat 1 ----------------------------------
\begin{frame}[fragile]\justifying
\begin{itemize}
\item[1.]\justifying (\`{E}psilon de m\`{a}quina). Es defineix el concepte de
	``\`{e}psilon de m\`{a}quina'' com el nombre positiu $\epsilon$ 
	m\'{e}s	petit que sumat a $1$ d\'{o}na diferent de $1$. \'{E}s a dir:
    	\begin{displaymath}
      		\epsilon:=\min\{\varepsilon > 0: 
			\mathrm{fl}(1+\varepsilon) > 1\}.
    	\end{displaymath}
	calculeu l'\`{e}psilon de m\`{a}quina tant en precisi\'{o} simple
    	(\texttt{float}) com en precisi\'{o} doble (\texttt{double}). 
\end{itemize}
\begin{rem}\justifying
En \texttt{C/C++} podem conèixer directament l'èpsilon de la màquina amb les 
constants \texttt{FLT\_EPSILON} (per a precisió simple) i \texttt{DBL\_EPSILON} 
(per a precisió) doble. Totes dues definides a la llibreria \texttt{cfloat}:
\begin{lstlisting}
...
#include<cfloat>
...
cout << "eps (single) = " << FLT_EPSILON << endl;
cout << "eps (double) = " << DBL_EPSILON << endl;
...
\end{lstlisting}
\end{rem}
\end{frame}

%------------------------------- Activitat 2 ----------------------------------
\begin{frame}\justifying
\begin{itemize}
\item[2.]\justifying 
	(Exercici 4 de la co\lgem ecció: suma de la s\`{e}rie harm\`{o}nica 
	generalitzada). Useu arit\-m\`{e}\-ti\-ca de 3 d\'{\i}gits amb 
	eliminaci\'{o} per a calcular la suma
   	\begin{displaymath}
    		S_{15}=\sum_{n=1}^{15}\frac{1}{n^{2}}, 
    	\end{displaymath}
    	primer en `'l'ordre natural'' (decreixent), i.e.,
    	\begin{displaymath}
    		S_{15}=\frac{1}{1} + \frac{1}{4}
    			+ \dots 
    			+ \frac{1}{255},
  	\end{displaymath}
  	i despr\'{e}s en l'ordre ``invers'' (creixent), i.e.,
  	\begin{displaymath}
    		S_{15}=\frac{1}{255}+ \frac{1}{196} + \dots
    			+ \frac{1}{1}.
  	\end{displaymath}
    	Decidiu quin \'{e}s
    	el m\`{e}tode m\'{e}s exacte de tots dos. \emph{Nota:} recordeu 
	l'exercici $3$. 
\end{itemize}
\end{frame}

%------------------------------- Activitat 3 ----------------------------------
\begin{frame}\justifying
\begin{itemize}\justifying	
\item[3.] (Pot\`{e}ncies del rec\'{\i}proc del nombre d'or). Sigui $\Phi =
      \frac{\sqrt{5}+1}{2}$ el nombre d'or (o proporci\'{o}, o ra\'{o}
      \`{a}uria) i $\phi$ el seu rec\'{\i}proc (invers multiplicatiu), 
      aix\`{o}  \'{e}s: $\phi = \frac{\sqrt{5} -1}{2}$. Es comprova d'immediat 
      que $-\Phi$ i $\phi$ s\'{o}n les dues arrels de l'equaci\'{o} de 
      2on.~grau
      \begin{displaymath}
        	x^{2}+x-1 = 0.
      \end{displaymath}
      Basant-vos en aix\`{o}, justifiqueu els tres algorismes seg\"{u}ents
\begin{enumerate}
\item[A1:]\justifying
  \begin{displaymath}
    \phi^{0}=1,\quad \phi^{1}= \phi,\quad \phi^{k}=\phi\cdot\phi^{k-1},
   \tag{\mbox{per a $k=2,3,\dots$}} 
    \end{displaymath}
\item[A2:]
  \begin{displaymath}
  \phi^{0}=1,\quad \phi^{1}= \phi,\quad \phi^{2} = 1-\phi,\quad
  \phi^{k} = \phi^{k-2}\cdot(1-\phi),
  \tag{\mbox{per a $k=3,4,\dots$}} 
\end{displaymath}
\item[A3:]
  \begin{displaymath}
  \phi^{0}=1,\quad \phi^{1}= \phi,\quad \phi^{2} = 1-\phi,\quad
  \phi^{k} = \phi^{k-2}-\phi^{k-1},
   \tag{\mbox{per a $k=3,4,\dots$}} 
\end{displaymath}
\end{enumerate}
per a calcular les pot\`{e}ncies successives de $\phi$. Escriviu un 
programa $\texttt{C/C++}$ que compari tots tres algorismes. 
\end{itemize}
\end{frame}

%------------------------------- Activitat 4 ----------------------------------
\begin{frame}\justifying
\begin{itemize}	
\item[4.] (Exercici 9 de la co\lgem ecció). Usant un m\`{e}tode recurrent, 
	calculeu el valor de les integrals,
	\begin{displaymath}
        	J_{k} = \int_{0}^{1} x^{k}\sin(\pi x)\rmd x,
      	\end{displaymath}
      	($j = 2,4,\dots,20$). En concret:
\begin{enumerate}[4.1]\justifying
\item Demostreu que:
\begin{equation}\label{eq:Jk}
\begin{split}
   J_{0} &= \frac{2}{\pi},\\
   J_{1} &=\frac{1}{\pi},\\ 
   J_{k} &= \frac{1}{\pi}-\frac{k(k-1)}{\pi^{2}}J_{k-2},\qquad
            k=2,3,4,\dots
\end{split}
\end{equation}
\item Fent servir la recurr\`{e}ncia~\eqref{eq:Jk}, calcueu
      $J_{k}$ per a $k = 2,4,6\dots,20$. \'{E}s estable l'algorisme
	trobat? Per qu\`{e}? \textbf{Nota:} direm que un algorisme és
	\emph{estable} si els errors en les aproximacions
	s'esmorteeixen al llarg de les iteracions. Si, pel contrari,
	 s'amplifiquen, direm que és \emph{inestable}.
\item Dissenyeu un algorisme alternatiu que eviti els problemes
       detectats al punt anterior. \emph{Ajut:} considereu~\eqref{eq:Jk}
\end{enumerate} 
\end{itemize}
\end{frame}

%------------------------------- Activitat 5 ----------------------------------
\begin{frame}[shrink=2]
\vspace{\baselineskip}
\begin{itemize}
\item[5.]\justifying 
	 Feu un programa en \texttt{C/C++} que avalu\"{\i} el polinomi de 
	 McLaurin $p_N(x)$ de grau $N$ de la funci\'o $f(x)=\rme^x$ en 
	 un punt donat $x$. Empreu la \emph{regla de Horner}.
\end{itemize}
\end{frame}

\begin{frame}%\justifying
\vspace{\baselineskip}
\begin{itemize}
\item[6.]\justifying (Suma d'una sèrie alternada) Volem comparar les
	aproximacions de $\rme^{-2}$ amb tres m\`etodes	diferents:
\begin{enumerate}[6.1]%[\sl ({\ref{exer:serieAlt}}.1)] 
\item\justifying Fent servir la funci\'o {\tt exp(x)} proporcionada a la
		 llibreria {\tt math.h} de \texttt{C}.  
\item\justifying Aproximant $\rme^{-2}$ per $p_{10}(-2)$, programat al 
		problema $5$.  
\item Usant $q_{10}(-2):=1/p_{10}(2)$.
\end{enumerate}
Aleshores:
\begin{itemize}%\justify
\item\justify Escriviu un programa en \texttt{C/C++} que calculi i
		tregui per pantalla (o b\'{e} escrigui a un fitxer) el 
		valor de les tres aproximacions:
\begin{displaymath}
	\texttt{exp(-2)},\qquad p_{10}(-2),\qquad q_{10}(-2)
\end{displaymath}
		i de les estimacions dels corresponents errors absoluts 
		i relatius, donades respectivament per
\begin{align*}
	\varepsilon_{a}\left(\texttt{exp(-2)}\right)&\approx
   	\bar{p}(-2)-\texttt{exp(-2)},\\
	\varepsilon_{r}\left(\text{\texttt{exp(-2)}}\right)&\approx
    		\frac{\varepsilon_{a}\left(\texttt{exp(-2)}\right)}
    		{\texttt{exp(-2)}},
\end{align*}
		on $\bar{p} = p_{10}$, $q_{10}$. Amb aquest resultats 
		completeu la Taula~\ref{taula:exp}. Feu-lo tant per  
		precisi\'o simple (\texttt{float}) com per doble 
		(\texttt{double}).
\end{itemize}
\begin{itemize}
\item Feu el mateix amb $\rme^{-7}$. Comenteu els resultats en
	ambd\'os casos.
\end{itemize}
\end{itemize}
\end{frame}

%------------------------------- Activitat 6 ----------------------------------
\begin{frame}[shrink=2]
\renewcommand{\arraystretch}{1,5}
\begin{table}[!t]
\begin{tabu} to \textwidth {|c|X[c]||X[c]|X[c]|}\hline
\texttt{exp(-2)} &         &  $e_{a}$    &   $e_{r}$ \\ \hline
$p_{10}(-2)$      &         &            &           \\ \hline
$q_{10}(-2)$      &         &            &           \\\hline
\end{tabu}
\caption%
{Comparaci\'o en el c\`alcul d'aproximacions de $\rme^{-2}$.}%
\label{taula:exp}
\end{table}
\end{frame}

\end{document}

%%% Local Variables:
%%% mode: latex
%%% TeX-master: t
%%% End:
